%!TEX encoding = UTF-8 Unicode
% -*- coding: UTF-8; -*-
% vim: set fenc=utf-8
\documentclass[12pt, ULlof, ULlot]{ULrapport}

\usepackage[utf8]{inputenc}
\usepackage[autolanguage]{numprint}
\usepackage[french]{babel}
\usepackage{icomma}
\usepackage{graphicx}
\usepackage{amsmath}
\usepackage{hyperref}
\usepackage{multirow}
\usepackage{array}
\usepackage{afterpage}
\usepackage{longtable}
\usepackage{booktabs}

\makeatletter
\newcommand{\thickhline}{%
    \noalign {\ifnum 0=`}\fi \hrule height 1pt
    \futurelet \reserved@a \@xhline
}
\newcommand{\e}[1]{\ensuremath{\times 10^{#1}}}
\newcommand{\degree}{\ensuremath{^\circ}}
\newcolumntype{"}{@{\hskip\tabcolsep\vrule width 1pt\hskip\tabcolsep}}
\makeatother

%Cette ligne permet de changer le nom d'un tableau : il change «Table x.x» (anglais) pour «Tableau x.x» (français). La classe ULrapport est suppose le faire, mais ne le fait pas.
\addto\captionsfrench{\def\tablename{{\sc{Tableau}}}}

\setlength\parskip{\baselineskip}

\TitreProjet{VisuaLigue (VL)}
\TitreRapport{Rapport de projet -- version 1.0}
\Destinataire{Martin Savoie, Jonathan Gaudreault}
\NumeroEquipe{01}
\NomEquipe{GIFTW}
\TableauMembres{%
        111\,099\,643 & Alexandre Gingras-Courchesne &\\\hline
        111\,106\,456 & Jérôme Isabelle &\\\hline
        111\,107\,781 & Maxime Ménard &\\\hline
        111\,098\,395 & Alexandra Mercier &\\\hline
}

\DateRemise{2 octobre 2016}

\HistoriqueVersions{%
    0.0 & 18 septembre 2016 & Création du document \\\hline
    1.0 & 2 octobre 2016 & Première remise \\\hline
}

%Corps du document

\begin{document}

% Chapitres
%!TEX encoding = UTF-8 Unicode
% -*- coding: UTF-8; -*-
% vim: set fenc-utf-8

\chapter{Vision}
\label{s:vision}

Suite à une rencontre avec le président de l'AEMQ (Association des entraîneurs mineurs du Québec), notre \textbf{jeune pousse} s'est vu confier le mandat de réaliser l'application VisuaLigue, qui a pour objectif de simplifier l'enseignement des stratégies de jeu pour les entraîneurs.
Cette application doit permettre aux entraîneurs de créer facilement des stratégies, puis de les présenter de façon dynamique et en temps réel aux joueurs.

La cr\'eation des strat\'egies s'effectue en dessinant les s\'eries d'actions que les joueurs doivent ex\'ecuter.
Un entraîneur peut créer une nouvelle stratégie ou modifier une stratégie existante.
La création d'une stratégie nécessite que chaque joueur ait sa série d'actions.
Pour cela, il faut s\'electionner un joueur, lui assigner un r\^ole et dessiner ensuite les actions qu'il doit faire.
Quand une strat\'egie est compl\'et\'ee, il est possible de la sauvegarder et de l'exporter.
Un utilisateur, que ce soit un entraîneur ou un joueur, peut visualiser une strat\'egie pr\'ec\'edemment d\'efinie.
Lors de la visualisation, il est possible d'arr\^eter, de red\'emarrer, d'avancer ou de reculer le visionnement.
De plus, plusieurs sports peuvent \^etre d\'efinis.
L'entraîneur peut aussi créer un nouveau sport ou en modifier un existant.
Il peut ainsi sp\'ecifier un terrain, un projectile et des r\^oles pour un sport.
La création d'un terrain se fait de deux façons: en important une image représentant le terrain ou en dessinant les lignes via l'application.
Pour les deux cas, il faut que les dimensions du terrain soient spécifiées.

L'application permet d'obtenir des strat\'egies que l'utilisateur peut visualiser en temps r\'eel.
Puisque la visualisation des stratégies décrites par un entraîneur est le but principal, il est primordial que l\'utilisation de l\'application soit naturelle et que son interface ne soit pas surchargée.

%!TEX encoding = UTF-8 Unicode
% -*- coding: UTF-8; -*-
% vim: set fenc-utf-8
%

\chapter{Exigences}
\label{s:exigences}

%!TEX encoding = UTF-8 Unicode
% -*- coding: UTF-8; -*-
% vim: set fenc-utf-8
%
% Chapitre Modélisation métier
%

\chapter{Modélisation métier}
\label{s:model_metier}


% Annexes
\appendix
%!TEX encoding = UTF-8 Unicode
% -*- coding: UTF-8; -*-
% vim: set fenc-utf-8
%

\chapter{Glossaire}
\label{s:glossaire}

\begin{itemize}
    \item Action: Déplacement ou interaction avec le projectile d'un joueur.
        \\
    \item Entra\^ineur: Utilisateur qui construit une strat\'egie et la pr\'esente \`a des joueurs.
        \\
    \item Joueur: Utilisateur qui ex\'ecute une action et qui poss\`ede un r\^ole.
        \\
    \item Obstacle: Objet statique qu'un joueur doit \'eviter lors d'un d\'eplacement.
        \\
    \item Projectile: Objet principal d'un sport, \textit{e.g:} balle, ballon, rondelle, etc.
        \\
    \item R\^ole: Fonction d'un joueur lors d'une strat\'egie, \textit{e.g:} ailier, centre, d\'efenseur, etc.
        \\
    \item Stratégie: Ensemble des s\'equences des actions des joueurs.
        \\
\end{itemize}

%!TEX encoding = UTF-8 Unicode
% -*- coding: UTF-8; -*-
% vim: set fenc-utf-8
%

% Ordre alphabétique selon la ref bibitem
\begin{thebibliographyUL}{99}
    %Enonce de projet
    \bibitem{enonce} Jonathan Gaudreault, Martin Savoie, 2016, \emph{Énoncé de projet: VisuaLigue}, Département d'informatique et de génie logiciel de l'Université Laval, 4 p.
\end{thebibliographyUL}

\end{document}
% Fin du document
