%!TEX encoding = UTF-8 Unicode
% -*- coding: UTF-8; -*-
% vim: set fenc=utf-8
\documentclass[12pt, ULlof, ULlot]{ULrapport}

\usepackage[utf8]{inputenc}
\usepackage[autolanguage]{numprint}
\usepackage[french]{babel}
\usepackage{icomma}
\usepackage{graphicx}
\usepackage{amsmath}
\usepackage{hyperref}
\usepackage{multirow}
\usepackage{array}
\usepackage{afterpage}
\usepackage{longtable}
\usepackage{booktabs}

\makeatletter
\newcommand{\thickhline}{%
    \noalign {\ifnum 0=`}\fi \hrule height 1pt
    \futurelet \reserved@a \@xhline
}
\newcommand{\e}[1]{\ensuremath{\times 10^{#1}}}
\newcommand{\degree}{\ensuremath{^\circ}}
\newcolumntype{"}{@{\hskip\tabcolsep\vrule width 1pt\hskip\tabcolsep}}
\makeatother

%Cette ligne permet de changer le nom d'un tableau : il change «Table x.x» (anglais) pour «Tableau x.x» (français). La classe ULrapport est suppose le faire, mais ne le fait pas.
\addto\captionsfrench{\def\tablename{{\sc{Tableau}}}}

\setlength\parskip{\baselineskip}

\TitreProjet{VisuaLigue (VL)}
\TitreRapport{Rapport de projet -- version 1.0}
\Destinataire{Martin Savoie, Jonathan Gaudreault}
\NumeroEquipe{01}
\NomEquipe{GIFTW}
\TableauMembres{%
        111\,099\,643 & Alexandre Gingras-Courchesne &\\\hline
        111,XXX,XXX & Jérôme Isabelle &\\\hline
        111,XXX,XXX & Maxime Ménard &\\\hline
        111\,098\,395 & Alexandra Mercier &\\\hline
}

\DateRemise{2 octobre 2016}

\HistoriqueVersions{%
    0.0 & 18 septembre 2016 & création du document \\\hline
}

%Corps du document

\begin{document}

% Chapitres
%!TEX encoding = UTF-8 Unicode
% -*- coding: UTF-8; -*-
% vim: set fenc-utf-8
%

\chapter{Introduction}
\label{s:model_metier}

%!TEX encoding = UTF-8 Unicode
% -*- coding: UTF-8; -*-
% vim: set fenc-utf-8
%

\chapter{Exigences}
\label{s:exigences}

%!TEX encoding = UTF-8 Unicode
% -*- coding: UTF-8; -*-
% vim: set fenc-utf-8
%
% Chapitre Modélisation métier
%

\chapter{Modélisation métier}
\label{s:model_metier}


% Annexes
\appendix
%!TEX encoding = UTF-8 Unicode
% -*- coding: UTF-8; -*-
% vim: set fenc-utf-8
%

\chapter{Glossaire}
\label{s:glossaire}

\begin{itemize}
    \item Action: Déplacement ou interaction avec le projectile d'un joueur.
        \\
    \item Entra\^ineur: Utilisateur qui construit une strat\'egie et la pr\'esente \`a des joueurs.
        \\
    \item Joueur: Utilisateur qui ex\'ecute une action et qui poss\`ede un r\^ole.
        \\
    \item Obstacle: Objet statique qu'un joueur doit \'eviter lors d'un d\'eplacement.
        \\
    \item Projectile: Objet principal d'un sport, \textit{e.g:} balle, ballon, rondelle, etc.
        \\
    \item R\^ole: Fonction d'un joueur lors d'une strat\'egie, \textit{e.g:} ailier, centre, d\'efenseur, etc.
        \\
    \item Stratégie: Ensemble des s\'equences des actions des joueurs.
        \\
\end{itemize}

%!TEX encoding = UTF-8 Unicode
% -*- coding: UTF-8; -*-
% vim: set fenc-utf-8
%

% Ordre alphabétique selon la ref bibitem
\begin{thebibliographyUL}{99}
    %Enonce de projet
    \bibitem{enonce} Jonathan Gaudreault, Martin Savoie, 2016, \emph{Énoncé de projet: VisuaLigue}, Département d'informatique et de génie logiciel de l'Université Laval, 4 p.
\end{thebibliographyUL}

\end{document}
% Fin du document
