%!TEX encoding = UTF-8 Unicode
% -*- coding: UTF-8; -*-
% vim: set fenc=utf-8
\documentclass[12pt, ULlof, ULlot]{ULrapport}

\usepackage[utf8]{inputenc}
\usepackage[autolanguage]{numprint}
\usepackage[french]{babel}
\usepackage{icomma}
\usepackage{graphicx}
\usepackage{amsmath}
\usepackage{hyperref}
\usepackage{multirow}
\usepackage{array}
\usepackage{afterpage}
\usepackage{longtable}
\usepackage{booktabs}

\makeatletter
\newcommand{\thickhline}{%
    \noalign {\ifnum 0=`}\fi \hrule height 1pt
    \futurelet \reserved@a \@xhline
}
\newcommand{\e}[1]{\ensuremath{\times 10^{#1}}}
\newcommand{\degree}{\ensuremath{^\circ}}
\newcolumntype{"}{@{\hskip\tabcolsep\vrule width 1pt\hskip\tabcolsep}}
\makeatother

%Cette ligne permet de changer le nom d'un tableau : il change «Table x.x» (anglais) pour «Tableau x.x» (français). La classe ULrapport est suppose le faire, mais ne le fait pas.
\addto\captionsfrench{\def\tablename{{\sc{Tableau}}}}

\setlength\parskip{\baselineskip}

\TitreProjet{VisuaLigue (VL)}
\TitreRapport{Rapport de projet -- version 2.0}
\Destinataire{Martin Savoie, Jonathan Gaudreault}
\NumeroEquipe{01}
\NomEquipe{GIFTW}
\TableauMembres{%
        111\,099\,643 & Alexandre Gingras-Courchesne &\\\hline
        111\,106\,456 & Jérôme Isabelle &\\\hline
        111\,107\,781 & Maxime Ménard &\\\hline
        111\,098\,395 & Alexandra Mercier &\\\hline
}

\DateRemise{6 novembre 2016}

\HistoriqueVersions{%
    0.0 & 18 septembre 2016 & Création du document \\\hline
    1.0 & 2 octobre 2016 & Première remise \\\hline
    2.0 & 6 novembre 2016 & Livrable 3\\\hline
}

%Corps du document

\begin{document}

% Chapitres
%!TEX encoding = UTF-8 Unicode
% -*- coding: UTF-8; -*-
% vim: set fenc-utf-8

\chapter{Diagrammes de classe de conception}
\label{s:classe_conception}

Ce chapitre\ldots

%!TEX encoding = UTF-8 Unicode
% -*- coding: UTF-8; -*-
% vim: set fenc-utf-8

\chapter{Diagramme d'activité}
\label{s:diagramme_activite}

\begin{figure}[htpb]
    \centering
    \includegraphics[scale=0.60]{fig/activity_diagram_imagepimage.png}
    \caption{Diagramme d'activité de la création d'une stratégie en mode image par image}
    \label{fig:activity_diagram_imagepimage}
\end{figure}

La figure \ref{fig:activity_diagram_imagepimage} représente le diagramme d'activité lors de l'édition de la stratégie en mode image par image.


%!TEX encoding = UTF-8 Unicode
% -*- coding: UTF-8; -*-
% vim: set fenc-utf-8

\chapter{Diagrammes d'état}
\label{s:diagrammes_etats}

\section{Diagramme d'état d'un joueur}
\label{sec:diagramme_etat_joueur}

\begin{figure}[htpb]
    \centering
    \includegraphics[scale=0.32]{fig/state_diag_player.png}
    \caption{Diagramme d'état d'un joueur}
    \label{fig:state_diag_player}
\end{figure}

La figure \ref{fig:state_diag_player} représente le diagramme d'état d'un joueur lors de l'édition de la stratégie.
Au démmarage de la simulation, les joueurs sont inactifs.
Cela est équivalent à l'idée que les joueurs sont << sur le banc >>.
Placer les joueurs sur le terrain les mets en mode actif.
Lorsqu'un des joueurs a l'état << hasProjectile >>, le projectile change d'état, tel que démontré dans la figure \ref{fig:state_diag_projectile}.
Aussi, le nombre maximum de joueur pouvant avoir cet état  en même temps est limité par le nombre de projectiles sur le terrain.
En mode d'édition image par image, créer un nouvelle image mets tous les joueurs sur le terrain en mode transparent.
Le joueur revient à son état précédent si l'utilisateur fait une action sur le joueur.

\section{Diagramme d'état d'un projectile}
\label{sec:diagramme_etat_projectile}

\begin{figure}[htpb]
    \centering
    \includegraphics[scale=0.32]{fig/state_diag_projectile.png}
    \caption{Diagramme d'état d'un projectile}
    \label{fig:state_diag_projectile}
\end{figure}

La figure \ref{fig:state_diag_projectile} représente le diagramme d'état d'un projectile lors de l'édition d'une stratégie.
Le projectile sera inactif si l'application n'est pas en mode édition, que ce soit image par image ou temps réel.
L'état de la gestion de la collision doit respecter deux conditions.
Il faut que le projectile << colle >> un obstacle, soit que la distance entre l'obstacle et le projectile est plus petit que le rayon de l'obstacle.
De plus, il faut que l'obstacle ait l'option << gère les collisions >>.
Sinon, le projectile se déplace comme si l'obstacle n'existait pas.


%!TEX encoding = UTF-8 Unicode
% -*- coding: UTF-8; -*-
% vim: set fenc-utf-8

\chapter{Diagrammes de séquence de conception}
\label{s:sequence_conception}

Ce chapitre ...


% Annexes
\appendix
%!TEX encoding = UTF-8 Unicode
% -*- coding: UTF-8; -*-
% vim: set fenc-utf-8
%

\chapter{Échéancier}
\label{s:echeancier}

L'image de l'\'ech\'eancier est disponible sous \textit{/rapport/fig/echeancier.png}.

\begin{figure}[htpb]
    \centering
    \includegraphics[scale=0.29, angle=90]{fig/echeancier.png}
    \caption{Gantt pour la planification du projet}
    \label{fig:echeancier}
\end{figure}

%!TEX encoding = UTF-8 Unicode
% -*- coding: UTF-8; -*-
% vim: set fenc-utf-8

\chapter{Vision}
\label{s:vision}

Suite à une rencontre avec le président de l'AEMQ (Association des entraîneurs mineurs du Québec), notre équipe s'est vu confier le mandat de réaliser l'application VisuaLigue, qui a pour objectif de simplifier l'enseignement des stratégies de jeu pour les entraîneurs.
Cette application doit permettre aux entraîneurs de créer facilement des stratégies, puis de les présenter de façon dynamique et en temps réel.

La cr\'eation des strat\'egies s'effectue en dessinant les s\'eries d'actions que les joueurs doivent ex\'ecuter.
Un entra\^ineur s\'electionne un joueur, lui assigne un r\^ole et dessine ensuite les actions qu'il doit faire.
Ceci est r\'ep\'et\'e pour tous les joueurs.
Quand une strat\'egie est compl\'et\'ee il est possible de la sauvegarder et de l'exporter.
Un utilisateur peut \`a sa guise visualiser une strat\'egie pr\'ec\'edemment d\'efinie.
Lors de la visualisation, il est possible d'arr\^eter, de red\'emarrer, d'avancer ou de reculer le visionnement.
De plus, plusieurs sports peuvent \^etre d\'efinies et ce par l'utilisateur.
L'ajout d'un sport ce fait en sp\'ecifiant un terrain, un projectile et des r\^oles.

L'application permet d'obtenir des strat\'egies que l'utilisateur peut visualiser en temps r\'eel.
Puisque la visualisation des stratégies décrites par un entraîneur est le but principal, il est primordial que son utilisation soit naturelle et que son interface ne soit pas surchargée.

%!TEX encoding = UTF-8 Unicode
% -*- coding: UTF-8; -*-
% vim: set fenc-utf-8

\chapter{Maquette des interfaces}
\label{s:maquettes_interfaces}

Ce chapitre présente les maquettes des principales interfaces de l'application VisuaLigue.
Une courte explication des fonctionnalités moins évidentes est également présente à la suite des figures si nécessaire.

\begin{figure}[htpb]
    \centering
    \includegraphics[scale=0.6]{fig/gui/creationStackPane.png}
    \caption{Interface pour la création des stratégies}
    \label{fig:gui:creationStackPane}
\end{figure}

L'interface \ref{fig:gui:creationStackPane} permet l'édition des stratégies, soit en mode image par image ou en mode temps réel.
Elle apparait lorsque l'on appuie sur le bouton temps réel ou image par image.
Un menu à gauche permet de glisser des joueurs, obstacles ou projectiles sur le terrain.
Un menu à droite permet d'ajouter une équipe dans la liste des joueurs à gauche.
Il permet aussi de choisir si on affiche le rôle et le nom des joueurs sur le terrain.
Si on clique sur un joueur sur le terrain, le menu à droite change pour éditer son nom, son rôle et son orientation et une case est présente pour décider si le joueur possède le projectile. 
Les icônes "suivant" et "précédent" permettent respectivement d'avancer ou de reculer d'une image dans le mode image par image.
Les étiquettes "x" et "y" dans le bas de la fenêtre correspondent aux coordonnées de la souris sur le terrain en unités réelles.
La transparence de certains joueurs sur la figure ci-dessus correspond à des éléments qui proviennent d'une image précédente lors de l'édition en mode image par image.
Le groupe de boutons en haut à droite permet d'effectuer diverses actions par rapport à l'édition de la stratégie. 
Le groupe du milieu permet de changer le mode.
Le groupe de gauche permet des actions plus générales.

\begin{figure}[htpb]
    \centering
    \includegraphics[scale=0.6]{fig/gui/mediaContent.png}
    \caption{Interface pour la visualisation des stratégies}
    \label{fig:gui:mediaContent}
\end{figure}

L'interface \ref{fig:gui:mediaContent} permet la visualisation des stratégies préalablement créées.
Elle apparait lorsque l'on appuie sur le bouton visionner.
En plus des boutons habituels de visionnement de vidéos, on y retrouve un curseur glissant permettant de se déplacer facilement à n'importe quel temps de la stratégie.
Il y a aussi une boîte de saisie et un bouton associé pour rejouer une séquence de la stratégie.
Les coordonnées de la souris sont encore affichées.

\begin{figure}[htpb]
    \centering
    \includegraphics[scale=0.6]{fig/gui/openStrategy.png}
    \caption{Interface de gestion des stratégies et des sports}
    \label{fig:gui:openStrategy}
\end{figure}

L'interface \ref{fig:gui:openStrategy} permet la gestion des stratégies et des sports.
Après avoir sélectionné une stratégie ou un sport, une multitude de choix s'offre à l'entraîneur.
Il peut supprimer un sport ou une stratégie, visionner une stratégie, ou bien l'éditer soit en mode temps réel ou en mode image par image.
\newpage

\begin{figure}[htpb]
    \centering
    \includegraphics[scale=0.6]{fig/gui/openObstacle.png}
    \caption{Interface de gestion des obstacles}
    \label{fig:gui:openObstacle}
\end{figure}

L'interface \ref{fig:gui:openObstacle} permet la gestion des obstacles.
On peut visualiser les types d'obstacles et en supprimer via cette interface.

\newpage

\begin{figure}[htpb]
    \centering
    \includegraphics[scale=0.6]{fig/gui/newStrategy.png}
    \caption{Interface de création des stratégies}
    \label{fig:gui:newStrategy}
\end{figure}

La fenêtre de dialogue \ref{fig:gui:newStrategy} permet de créer une stratégie.
Elle apparaît lorsque l'on appuie sur le bouton nouveau/stratégie.
Tous les paramètres nécessaires à la création d'une stratégie y sont présents.

\begin{figure}[htpb]
    \centering
    \includegraphics[scale=0.6]{fig/gui/newSport.png}
    \caption{Interface de création des sports}
    \label{fig:gui:newSport}
\end{figure}

La fenêtre de dialogue \ref{fig:gui:newSport} permet de créer un sport.
Elle apparaît lorsque l'on appuie sur le bouton nouveau/sport.
Tous les paramètres nécessaires à la création d'une stratégie y sont présents, y compris la configuration du terrain, du projectile et des rôles.

\newpage

\begin{figure}[htpb]
    \centering
    \includegraphics[scale=0.6]{fig/gui/newObstacle.png}
    \caption{Interface de création des obstacles}
    \label{fig:gui:newObstacle}
\end{figure}

La fenêtre de dialogue \ref{fig:gui:newObstacle} permet de créer un sport.
Elle apparaît lorsque l'on appuie sur le bouton nouveau/obstacle.
Tous les paramètres nécessaires à la création d'un obstacle y sont présents, y compris un champs pour activer les collisions ou non sur l'obstacle.

\newpage
%!TEX encoding = UTF-8 Unicode
% -*- coding: UTF-8; -*-
% vim: set fenc-utf-8

\chapter{Modèle du domaine}
\label{s:modele_domaine}

Ce chapitre présente la modélisation du domaine avec un diagramme des classes conceptuelles de l'application.
La figure \ref{fig:conceptuel_diag} représente la modélisation du domaine.

\begin{figure}[htpb]
    \centering
    \includegraphics[scale=0.6]{fig/conceptuel_diag.png}
    \caption{Modélisation du modèle pour l'application}
    \label{fig:conceptuel_diag}
\end{figure}

%!TEX encoding = UTF-8 Unicode
% -*- coding: UTF-8; -*-
% vim: set fenc-utf-8

\chapter{Cas d'utilisations}
\label{s:cas_utilisation}

Ce chapitre présente les différents cas d'utilisation pour l'application VisuaLigue.
La figure \ref{fig:cas_utilisation_diag} résume les acteurs du systèmes et les cas d'utilisations.
La suite du chapitre décrie en détails les cas d'utilisations et s'attarde sur les plus importants.

\begin{figure}[htpb]
    \centering
    \includegraphics[scale=0.7]{fig/cas_utilisation_diag.png}
    \caption{Diagramme des cas d'utilisations}
    \label{fig:cas_utilisation_diag}
\end{figure}

\newpage



\section{Ajouter un type de sport}
\label{sec:ajouter_un_type_de_sport}

\begin{itemize}
    \item \textbf{Cas d'utilisation:} Ajouter un type de sport
    \item \textbf{Syst\`eme:} VisuaLigue
    \item \textbf{Acteur principal:} Entra\^ineur
    \item \textbf{Sc\'enario principal:}
        \begin{enumerate}
            \item L'entra\^ineur cr\'ee un sport, le nomme.
            \item Ensuite il ajoute les r\^oles du sport.
            \item Finalement, il d\'efini un terrain en dessinant les lignes et en sp\'ecifiant les dimensions.
        \end{enumerate}
    \item \textbf{Autres situations:}
    \begin{itemize}
        \item \textbf{Sport d\'ej\`a existant:} Si le sport existe d\'ej\`a dans l'application, un message d'avertissement appara\^it pour signaler que le sport existe d\'ej\`a.
        L'entraîneur peut d\'ecider d'effacer ce qui \'etait dans le sport existant, d'enregister son sport sous un autre nom, ou d'oublier le sport cr\'e\'e.
    \end{itemize}
\end{itemize}



\section{Ajouter une strat\'egie}
\label{sec:ajouter_une_strat'egie}
\begin{itemize}
    \item \textbf{Cas d'utilisation:} Ajouter une strat\'egie
    \item \textbf{Syst\`eme:} VisuaLigue
    \item \textbf{Acteur principal:} Entra\^ineur
    \item \textbf{Pr\'erequis:} Le sport pour lequel l'entraîneur veut ajouter une stratégie doit exister dans l'application.
    \item \textbf{Sc\'enario principal:}
        \begin{enumerate}
            \item L'entra\^ineur veut ajouter une nouvelle strat\'egie.
            \item Il lui assigne un identifiant et choisi un type de sport pour la strat\'egie.
        \end{enumerate}
    \item \textbf{Autres situations:}
        \begin{itemize}
            \item \textbf{Sport d\'ej\`a existant:}
            \begin{enumerate}
                \item Si la stratégie existe d\'ej\`a dans l'application, un message d'avertissement appara\^it.
                \item L'entra\^ineur a alors le choix entre \'ecraser la strat\'egie existante ou d'annuler son ajout.
            \end{enumerate}
        \end{itemize}
\end{itemize}



\section{\'Editer une stratégie}
\label{sec:ajouter_une_strategie}
\begin{itemize}
    \item \textbf{Cas d'utilisation:} \'Editer une strat\'egie
    \item \textbf{Syst\`eme:} VisuaLigue
    \item \textbf{Acteur principal:} Entra\^ineur
    \item \textbf{Pr\'erequis:}
    \item \textbf{Parties prenantes et int\'er\^ets:}
    \item \textbf{Garanties en cas de succ\`es:}
    \item \textbf{Sc\'enario principal:}
        \begin{enumerate}
            \item L'entraîneur assigne les r\^oles aux diff\'erents joueurs.
            \item Ensuite, il peut s\'electionner un joueur et tracer un mouvement ou interargir avec le projectile.
            \item Pendant que le joueur est s\'electionn\'e, une simulation en temps r\'eel des mouvements pr\'ec\'edemment d\'efinies s'ex\'ecute.
            \item L'entraîneur peut ensuite s\'electionner un autre joueur et dessiner une nouvelle action et la simulation recommence du d\'ebut.
        \end{enumerate}
    \item \textbf{Autres situations:}
        \begin{itemize}
            \item \textbf{Édition en mode image par image:} L'entraîneur édite la strat\'egie image par image ...
                \begin{enumerate}
                    \item foo
                    \item bar
                \end{enumerate}
        \end{itemize}
    \item \textbf{Fr\'equence d'utilisation:}
\end{itemize}



\section{Visualiser une stratégie}
\label{sec:visualiser_une_strategie}
\begin{itemize}
    \item \textbf{Cas d'utilisation:} Visualiser une strat\'egie
    \item \textbf{Syst\`eme:} VisuaLigue
    \item \textbf{Acteur principal:} Entra\^ineur ou joueur
    \item \textbf{Parties prenantes et int\'er\^ets:}
    \item \textbf{Pr\'erequis:} Il faut que la strat\'egie soit enregistr\'ee dans l'application pour que l'entraîneur puisse la visualiser.
    \item \textbf{Garanties en cas de succ\`es:}
    \item \textbf{Sc\'enario principal:}
        \begin{enumerate}
            \item L'entra\^ineur s\'electionne la strat\'egie \`a visualiser.
            \item Il d\'ebute la visualisation et observe le d\'eroulement de la strat\'egie.
            \item Il peut mettre fin \`a la visualisation \`a tout moment.
        \end{enumerate}
    \item \textbf{Autres situations:}
        \begin{itemize}
            \item fubar
                \begin{enumerate}
                    \item foo
                    \item bar
                \end{enumerate}
        \end{itemize}
\end{itemize}



\section{Sauvegarder une stratégie}
\label{sec:exporter_une_strategie}
\begin{itemize}
    \item \textbf{Cas d'utilisation:} Sauvegarder une strat\'egie
    \item \textbf{Syst\`eme:} VisuaLigue
    \item \textbf{Acteur principal:} Entra\^ineur
    \item \textbf{Sc\'enario principal:}
        \begin{enumerate}
            \item L'entra\^ineur a modifié une strat\'egie et la sauvegarde.
            \item L'application enregistre les \'el\'ements de la strat\'egie.
        \end{enumerate}
    \item \textbf{Autres situations:}
        \begin{itemize}
            \item \textbf{Exportation dans un format d'image:}
                \begin{enumerate}
                    \item L'entra\^ineur souhaite plut\^ot exporter la strat\'egie dans un format de fichier.
                    \item Il s\'electionne le format de fichier et le nom pour l'exportation.
                    \item L'application convertie la strat\'egie en une image et l'enregistre dans un fichier avec le bon nom.
                \end{enumerate}
        \end{itemize}
\end{itemize}



\section{Charger une strat\'egie}
\label{sec:charger_une_strat'egie}

\begin{itemize}
    \item \textbf{Cas d'utilisation:} Charger une strat\'egie
    \item \textbf{Syst\`eme:} VisuaLigue
    \item \textbf{Acteur principal:} Entra\^ineur
    \item \textbf{S\'ecnario principal:}
        \begin{enumerate}
            \item L'entra\^ineur souhaite \'editer une strat\'egie d\'ej\`a existante.
            \item Il s\'electionne la bonne strat\'egie puis la charge.
        \end{enumerate}
\end{itemize}

%!TEX encoding = UTF-8 Unicode
% -*- coding: UTF-8; -*-
% vim: set fenc-utf-8
%

\chapter{Glossaire}
\label{s:glossaire}

\begin{itemize}
    \item Action: Déplacement ou interaction avec le projectile effectué joueur.
        \\
    \item Entra\^ineur: Utilisateur qui construit une strat\'egie et la pr\'esente \`a des joueurs.
        \\
    \item Joueur: Utilisateur qui ex\'ecute une action et qui poss\`ede un r\^ole.
        \\
    \item Obstacle: Objet statique qu'un joueur doit \'eviter lors d'un d\'eplacement.
        \\
    \item Projectile: Objet principal d'un sport, \textit{e.g:} balle, ballon, rondelle, etc.
        \\
    \item R\^ole: Fonction d'un joueur lors d'une strat\'egie, \textit{e.g:} ailier, centre, d\'efenseur, etc.
        \\
    \item Stratégie: Ensemble des s\'equences des actions des joueurs.
        \\
\end{itemize}

%!TEX encoding = UTF-8 Unicode
% -*- coding: UTF-8; -*-
% vim: set fenc-utf-8
%

% Ordre alphabétique selon la ref bibitem
\begin{thebibliographyUL}{99}
    %Enonce de projet
    \bibitem{enonce} Jonathan Gaudreault, Martin Savoie, 2016, \emph{Énoncé de projet: VisuaLigue}, Département d'informatique et de génie logiciel de l'Université Laval, 4 p.
\end{thebibliographyUL}

\end{document}
% Fin du document
