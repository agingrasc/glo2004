%!TEX encoding = UTF-8 Unicode
% -*- coding: UTF-8; -*-
% vim: set fenc-utf-8

\chapter{Vision}
\label{s:vision}

Suite à une rencontre avec le président de l'AEMQ (Association des entraîneurs mineurs du Québec), notre équipe s'est vu confier le mandat de réaliser l'application VisuaLigue, qui a pour objectif de simplifier l'enseignement des stratégies de jeu pour les entraîneurs.
Cette application doit permettre aux entraîneurs de créer facilement des stratégies, puis de les présenter de façon dynamique et en temps réel.

La création des stratégies se fait en positionnant et en déplaçant les joueurs et le projectile sur la surface de jeu.
Une fois une stratégie complétée, l'application permet de sauvegarder celle-ci afin de pouvoir la réutiliser plus tard.
L'utilisateur peut à sa guise démarrer le visionnement d'une stratégie pour voir en temps réel la suite d'actions effectuées par les joueurs.
Au cours du visionnement, il a accès à divers contrôle, comme jouer, arrêter, avancer et reculer.
L'application VisuaLigue n'est pas restreinte à un seul sport.
Elle peut être utilisée pour tout sport d'équipe où les joueurs interagissent avec un projectile.
Il est d'ailleurs possible de créer différents types de sports avec des paramètres modifiables, tel que la configuration du terrain.

Le but de l'application étant de faciliter la visualisation des stratégies décrites par un entraîneur, il est primordial que son utilisation soit naturelle et que son interface ne soit pas surchargée.
