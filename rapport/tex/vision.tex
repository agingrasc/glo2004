%!TEX encoding = UTF-8 Unicode
% -*- coding: UTF-8; -*-
% vim: set fenc-utf-8

\chapter{Vision}
\label{s:vision}

Suite à une rencontre avec le président de l'AEMQ (Association des entraîneurs mineurs du Québec), notre "startup" s'est vu confier le mandat de réaliser l'application VisuaLigue, qui a pour objectif de simplifier l'enseignement des stratégies de jeu pour les entraîneurs.
Cette application doit permettre aux entraîneurs de créer facilement des stratégies, puis de les présenter de façon dynamique et en temps réel aux joueurs.

La cr\'eation des strat\'egies s'effectue en dessinant les s\'eries d'actions que les joueurs doivent ex\'ecuter.
Un entraîneur peut créer une nouvelle stratégie ou modifier une stratégie existante.
La création d'une stratégie nécessite que chaque joueur ait sa série d'actions.
Pour cela, il faut s\'electionner un joueur, lui assigner un r\^ole et dessiner ensuite les actions qu'il doit faire.
Quand une strat\'egie est compl\'et\'ee, il est possible de la sauvegarder et de l'exporter.
Un utilisateur, que ce soit un entraîneur ou un joueur, peut visualiser une strat\'egie pr\'ec\'edemment d\'efinie.
Lors de la visualisation, il est possible d'arr\^eter, de red\'emarrer, d'avancer ou de reculer le visionnement.
De plus, plusieurs sports peuvent \^etre d\'efinis.
L'entraîneur peut aussi créer un nouveau sport ou en modifier un existant.
Il peut ainsi sp\'ecifier un terrain, un projectile et des r\^oles pour un sport.
La création d'un terrain se fait de deux façons: en important une image représentant le terrain ou en dessinant les lignes via l'application.
Pour les deux cas, il faut que les dimensions du terrain soient spécifiées.

L'application permet d'obtenir des strat\'egies que l'utilisateur peut visualiser en temps r\'eel.
Puisque la visualisation des stratégies décrites par un entraîneur est le but principal, il est primordial que l'utilisation de l'application soit naturelle et que son interface ne soit pas surchargée.
