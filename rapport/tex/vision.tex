%!TEX encoding = UTF-8 Unicode
% -*- coding: UTF-8; -*-
% vim: set fenc-utf-8

\chapter{Vision}
\label{s:vision}

Suite à une rencontre avec le président de l'AEMQ (Association des entraîneurs mineurs du Québec), notre équipe s'est vu confier le mandat de réaliser l'application VisuaLigue, qui a pour objectif de simplifier l'enseignement des stratégies de jeu pour les entraîneurs.
Cette application doit permettre aux entraîneurs de créer facilement des stratégies, puis de les présenter de façon dynamique et en temps réel.

La cr\'eation des strat\'egies s'effectue en dessinant les s\'eries d'actions que les joueurs doivent ex\'ecuter.
Un entra\^ineur s\'electionne un joueur, lui assigne un r\^ole et dessine ensuite les actions qu'il doit faire.
Ceci est r\'ep\'et\'e pour tous les joueurs.
Quand une strat\'egie est compl\'et\'ee il est possible de la sauvegarder et de l'exporter.
Un utilisateur peut \`a sa guise visualiser une strat\'egie pr\'ec\'edemment d\'efinie.
Lors de la visualisation, il est possible d'arr\^eter, de red\'emarrer, d'avancer ou de reculer le visionnement.
De plus, plusieurs sports peuvent \^etre d\'efinies et ce par l'utilisateur.
L'ajout d'un sport ce fait en sp\'ecifiant un terrain, un projectile et des r\^oles.

L'application permet d'obtenir des strat\'egies que l'utilisateur peut visualiser en temps r\'eel.
Puisque la visualisation des stratégies décrites par un entraîneur est le but principal, il est primordial que son utilisation soit naturelle et que son interface ne soit pas surchargée.
