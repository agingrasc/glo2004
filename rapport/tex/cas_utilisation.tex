%!TEX encoding = UTF-8 Unicode
% -*- coding: UTF-8; -*-
% vim: set fenc-utf-8

\chapter{Cas d'utilisations}
\label{s:cas_utilisation}

\section{Ajouter un type de sport}
\label{sec:ajouter_un_type_de_sport}

\subsection{Sc\'enario principal}
\label{sub:sc'enario_principal}

L'\textit{entra\^ineur} veut définir un sport pour l'application.
Il cr\'ee un sport, le nomme.
Ensuite il ajoute les \textit{r\^oles} du sport.
Finalement, il d\'efini un terrain en dessinant les lignes et en sp\'ecifiant les dimensions.

\section{Ajouter une stratégie}
\label{sec:ajouter_une_strategie}

\subsection{Sc\'enario principal}
\label{sub:sc'enario_principal}

L'\textit{entraîneur} veut ajouter une nouvelle \textit{stratégie}.
Il choisi le sport pour la \textit{strat\'egie}.
Ensuite, pour chacun des joueurs, il leur assigne un \textit{r\^ole} et dessine leurs \textit{actions}.

\section{Visualiser une stratégie}
\label{sec:visualiser_une_strategie}

L'\textit{entra\^ineur} veut visualiser le d\'eroulement d'une \textit{strat\'egie}.
Il s\'electionne la \textit{strat\'egie} \`a visualiser.
Il d\'ebute la visualisation et observe le d\'eroulement de la \textit{strat\'egie}.
Il peut mettre fin \`a la visualisation \`a tout moment.

\section{Exporter une stratégie}
\label{sec:exporter_une_strategie}

L' \textit{entra\^ineur} veut exporter une strat\'egie.
Il s\'electionne la strat\'egie et le format pour l'exportation.
Il applique l'exportation de la \textit{strat\'egie}.
